\documentclass[10pt, aspectratio=169, progressbar=frametitle]{beamer}
\usetheme[numbering=fraction]{metropolis}

% -----------------------------------------------------------------------------
% Fonts
\usepackage[defaultsans,scale=1.0]{lato}
\usepackage{fourier}
\usepackage[scale=1.05]{inconsolata}
\usepackage{microtype}
\usefonttheme[onlymath]{serif}

% -----------------------------------------------------------------------------
% Typography
\usepackage{siunitx}

% -----------------------------------------------------------------------------
% Graphics
\usepackage{pgfplots}
\usepackage{tikz}
\usepackage[beamer,customcolors]{hf-tikz}
\usetikzlibrary{arrows.meta, calc, positioning, shapes.geometric}
\tikzset{%
  every path/.style={thick},
    edge/.style={%
        ->,
        >={Stealth[scale=1]},
    },
    usercode/.style={%
        rectangle,
        fill=orange!80,
        draw=black, thick,
        minimum width=6em,
        rounded corners=2pt,
        text=white,
        font=\bfseries,
        text centered,
    },
    lib/.style={%
        ellipse,
        fill=blue!80,
        minimum width=.16\linewidth,
        minimum height=.06\linewidth,
        rounded corners=2pt,
        text=white,
        font=\bfseries,
        text centered,
    },
    solver/.style={%
        rectangle,
        fill=green!80,
        draw=black, thick, dashed,
        text=white,
        font=\bfseries,
        text centered,
        minimum width=7em,
        rounded corners=2pt,
    },
}

\graphicspath{{./assets}}

% -----------------------------------------------------------------------------
% Define useful colors.
\colorlet{black}{black!50!gray}
\colorlet{green}{green!50!gray}
\colorlet{blue}{blue!50!gray}

\colorlet{CBlue}{blue!15}
\colorlet{CBlueD}{blue!95!gray}
\colorlet{CRed}{red!15}
\colorlet{CRedD}{red!65!gray}
\colorlet{CGreenD}{green!60!gray}
\colorlet{COrange}{orange}

% -----------------------------------------------------------------------------

\usepackage{minted}

% -----------------------------------------------------------------------------
\setbeamersize{text margin left=0.5cm, text margin right=0.5cm}

% -----------------------------------------------------------------------------
% Title info
\title{Open Interfaces for Scientific Computing}
\author{Dmitry I. Kabanov, Stephan Rave, Mario Ohlberger}
\institute{Instute for Analysis une Numerics, University of Münster}
\date{ESCO, June 2024, Pilsen}

% -----------------------------------------------------------------------------
\begin{document}
\maketitle

\begin{frame}{Typical problem}
  \begin{itemize}
    \item Student Floyd is mostly proficient in Python
    \item However, for the new project, he needs to run a Monte-Carlo
          simulation on a forward model implemented as a C solver
    \item Which means that he needs to write bindings to invoke the solver
    \item He learns about another solver and wants to try it
    \item \alert{Two problems:} different languages and different interfaces
  \end{itemize}
\end{frame}

\begin{frame}{Traditional way}
  \begin{minipage}{0.45\textwidth}
    \begin{itemize}
      \item For a solver in C provide bindings to Python, Julia, R, \dots, $L$ languages
      \item Repeat for a set of $I$ implementations of algorithms and $L$ languages
      \item It leads to $\mathcal O(L \times I)$ amount of work if we want to provide
            bindings for each algorithm in each language
    \end{itemize}
  \end{minipage}\hfill
  \begin{minipage}{0.45\textwidth}
    \begin{tikzpicture}
  \coordinate (left_side) at (-2, 3);
  \coordinate (right_side) at (2, 4.6);

  \node [usercode,] at (left_side) (user_py) {Python user};
  \node [usercode, below=of user_py] (user_c) {C user};
  \node [usercode, below=of user_c] (user_jl) {Julia user};

  \node [solver,] at (right_side) (solver1) {Python solver 1};
  \node [solver, below=of solver1] (solver2) {Python solver 2};
  \node [solver, below=of solver2] (solver3) {Julia solver};
  \node [solver, below=of solver3] (solver4) {C solver 1};
  \node [solver, below=of solver4] (solver5) {C solver 2};

  \draw[edge] (user_py) -- (solver1.west);
  \draw[edge] (user_py) -- (solver2);
  \draw[edge] (user_py) -- (solver3);
  \draw[edge] (user_py) -- (solver4);
  \draw[edge] (user_py) -- ($(solver5.north west) + (0.5cm, 0)$);

  \draw[edge] (user_c) -- (solver1.south west);
  \draw[edge] (user_c) -- (solver2);
  \draw[edge] (user_c) -- (solver3);
  \draw[edge] (user_c) -- (solver4);
  \draw[edge] (user_c) -- (solver5.north west);

  \draw[edge] (user_jl) -- ($(solver1.south west) + (0.5cm, 0)$);
  \draw[edge] (user_jl) -- (solver2);
  \draw[edge] (user_jl) -- (solver3);
  \draw[edge] (user_jl) -- (solver4);
  \draw[edge] (user_jl) -- (solver5.west);
\end{tikzpicture}

  \end{minipage}
\end{frame}

\begin{frame}{Another solution}
  \begin{itemize}
    \item To solve both of this problems:
          \begin{itemize}
            \item provide automatic bindings between different languages
            \item provide generic interfaces for typical numerical problems
            \item Examples of generic interfaces are:
                  \begin{itemize}
                    \item \texttt{PyMOR} for model-order reduction simulations
                    \item  \texttt{scipy.integrate} in Python
                          and \texttt{OrdinaryDiffEqs.jl} in Julia for solving initial-value problems for ODEs
                  \end{itemize}
          \end{itemize}
    \item However, the above packages provide solutions only
          to the multiple-interfaces problem, not to the problem
          of the multiple languages
  \end{itemize}
\end{frame}

\begin{frame}{MaRDI Open Interfaces}
  \begin{minipage}{0.45\textwidth}
    To solve both problems simultaneously:
    \begin{itemize}
      \item Provide a mediator library \texttt{OIF} that:
            \begin{itemize}
              \item handles data marshalling between different languages
              \item provides a set of generic interfaces
                    for typical numerical problems
            \end{itemize}
      \item Each interface abstracts out the descrepancies
            of different implementations
      \item For $L$ languages and $I$ implementations leads to
            the $\mathcal O (I + L)$ amount of work
    \end{itemize}
  \end{minipage}\hfill%
  \begin{minipage}{0.50\textwidth}
    \begin{tikzpicture}
  % \draw[help lines] (-2, -2) grid (5, 5);
  \coordinate (left_side) at (-2, 0);
  \coordinate (right_side) at (3, 0);
  \coordinate (center) at (0, 0);

  \node [usercode] at (left_side) (user_c) {C user};
  \node [usercode, above=of user_c] (user_py) {Python user};
  \node [usercode, below=of user_c] (user_jl) {Julia user};

  \node [lib] at (center) (liboif) {OIF};

  \node [solver] at (right_side) (solver3) {Julia solver};
  \node [solver, above=of solver3] (solver2) {Python solver 2};
  \node [solver, above=of solver2] (solver1) {Python solver 1};
  \node [solver, below=of solver3] (solver4) {C solver 1};
  \node [solver, below=of solver4] (solver5) {C solver 2};

  \draw[edge] (user_py) -- (liboif);
  \draw[edge] (user_c) -- (liboif);
  \draw[edge] (user_jl) -- (liboif);

  \draw[edge] (liboif) -- (solver1.west);
  \draw[edge] (liboif) -- (solver2.west);
  \draw[edge] (liboif) -- (solver3.west);
  \draw[edge] (liboif) -- (solver4.west);
  \draw[edge] (liboif) -- (solver5.west);
\end{tikzpicture}

  \end{minipage}
\end{frame}

\begin{frame}{Software architecture}
  \begin{center}
    \includegraphics[scale=0.09]{arch.png}
  \end{center}
\end{frame}

\begin{frame}{Currently supported languages, data types, problems}
  \begin{itemize}
    \item Languages: C, Python, Julia
    \item Data types:
          \begin{itemize}
            \item \texttt{OIF\_INT} - 32-bit integers
            \item \texttt{OIF\_FLOAT64} - 64-bit floating-point numbers
            \item \texttt{OIF\_ARRAY\_F64} - arrays of 64-bit floating-point numbers
            \item \texttt{OIF\_CALLBACK} - callback functions
            \item \texttt{OIF\_USER\_DATA} - user-data objects of volatile type
          \end{itemize}
    \item Interfaces:
          \begin{itemize}
            \item (Toy problem) quadratic equations: $ax^2 + bx + c = 0$
            \item systems of linear algebraic equations: $Ax = b$
            \item initial-value problems for ODEs \(y'(t) = f(t, y), \  y(t_0) = y_0\)
          \end{itemize}
  \end{itemize}

  {\small For C, special data structure \texttt{OIFArrayF64} is developed that packs
  together array's data, number of dimensions, and shape.
  This provides unified interfaces in different languages}
\end{frame}

\begin{frame}[fragile]{Interface for initial-value problems for ODEs}
  \begin{minipage}{\dimexpr0.22\textwidth-2\tabcolsep}
    \begin{align*}
      y'(t)  & = f(t, y) \\
      y(t_0) & = y_0
    \end{align*}
  \end{minipage}
  \begin{minipage}{\dimexpr0.73\textwidth-2\tabcolsep}
    \begin{minted}{Python}
      # Set initial value y(t0) = y0.
      set_initial_value(y0: np.ndarray, t0: float)

      # Specify right-hand side function f(t, y, ydot, user_data).
      set_rhs_fn(f: Callback)

      # Specify relative and absolute tolerances.
      set_tolerances(rtol: float, atol: float)

      # Integrate to time t and write solution to y.
      integrate(t: float, y: np.ndarray)

      # Set additional data passed to right-hand side function.
      set_user_data(user_data: object)
    \end{minted}
  \end{minipage}
\end{frame}

\begin{frame}{Performance study}
We solve a 2D Gray--Scott reaction-diffusion system\footnote{Pearson1993 EEEEXXXPAAAND!}:
\begin{align*}\label{eq:gs-system}
\begin{split}
\frac{\partial u}{\partial t} &= d_u \nabla^2 u - u v^2 + F (1 - u), \\
\frac{\partial v}{\partial t} &= d_v \nabla^2 v + u v^2 - (F + k) v
\end{split}
\end{align*}
with periodic boundary conditions on the domain $[-2.5; 2.5]^2$ with initial
condition given by \(u = 1\), \(v = 0\) everywhere in the domain, except
a square of \(40 \times 40\) grid points  centered in the center of the domain,
which is set to \(u = 0.5 + \mathrm{Uniform}(0; 0.1)\)
and \(v = 0.25 + \mathrm{Uniform}(0; 0.1)\).

Finatl time 100 with time step 1.
Parameters \(F = 0.055\), \(k = 0.062\), \(d_u = \num{2e-5}\),
\(d_v = \num{e-5}\).

\end{frame}

\begin{frame}{Performance study}

  \begin{minipage}{\dimexpr0.5\textwidth - 2\tabcolsep}
  Resolution $N \in \{64, 128, 256, 512\}$.

  Comparison: Using Sundials CVODE solver via Open Interfaces from Python
  vs direct bindings using \texttt{scikit.odes}\footnote{SCIKIT.ODES!!!}.
  \end{minipage}%
  \begin{minipage}{\dimexpr0.5\textwidth - 2\tabcolsep}
  \centering
  \includegraphics[scale=0.45]{ivp_cvode_gs_perf_normalized}
  \end{minipage}
\end{frame}

\begin{frame}[standout]{}
  \centering
  \huge{Thank you!}
\end{frame}

\end{document}
